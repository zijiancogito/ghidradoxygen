\hypertarget{coreclasses_coreintro}{}\section{Introduction}\label{coreclasses_coreintro}
The decompiler attempts to translate from low-\/level representations of computer programs into high-\/level representations. Thus it needs to model concepts from both the low-\/level machine hardware domain and from the high-\/level software programming domain.

Understanding the classes within the source code that implement these models provides the quickest inroad into obtaining an overall understanding of the code.

We list all these fundemental classes here, loosely grouped as follows. There is one set of classes that describe the {\itshape Syntax} {\itshape Trees}, which are built up from the original p-\/code, and transformed during the decompiler\textquotesingle{}s simplification process. The {\itshape Translation} classes do the actual building of the syntax trees from binary executables, and the {\itshape Transformation} classes do the actual work of transforming the syntax trees. Finally there is the {\itshape High-\/level} classes, which for the decompiler represents recovered information, describing familiar software development concepts, like datatypes, prototypes, symbols, variables, etc.

{\bfseries{Syntax}} {\bfseries{Trees}} 
\begin{DoxyItemize}
\item \mbox{\hyperlink{class_addr_space}{Addr\+Space}}
\begin{DoxyItemize}
\item A place within the reverse engineering model where data can be stored. The typical address spaces are {\bfseries{ram}}, modeling the main databus of a processor, and {\bfseries{register}}, modeling a processors on board registers. Data is stored a byte at a time at {\bfseries{offsets}} within the \mbox{\hyperlink{class_addr_space}{Addr\+Space}}.
\end{DoxyItemize}
\item \mbox{\hyperlink{class_address}{Address}}
\begin{DoxyItemize}
\item An \mbox{\hyperlink{class_addr_space}{Addr\+Space}} and an offset within the space forms the \mbox{\hyperlink{class_address}{Address}} of the byte at that offset.
\end{DoxyItemize}
\item \mbox{\hyperlink{class_varnode}{Varnode}}
\begin{DoxyItemize}
\item A contiguous set of bytes, given by an \mbox{\hyperlink{class_address}{Address}} and a size, encoding a single value in the model. In terms of S\+SA syntax tree, a \mbox{\hyperlink{class_varnode}{Varnode}} is also a node in the tree.
\end{DoxyItemize}
\item \mbox{\hyperlink{class_seq_num}{Seq\+Num}}
\begin{DoxyItemize}
\item A {\itshape sequence} {\itshape number} that extends \mbox{\hyperlink{class_address}{Address}} for distinguishing Pcode\+Ops describing a single instruction.
\item \mbox{\hyperlink{coreclasses_classseqnum}{Overview of Seq\+Num}}
\end{DoxyItemize}
\item \mbox{\hyperlink{class_pcode_op}{Pcode\+Op}}
\begin{DoxyItemize}
\item A single {\itshape p-\/code} operation. A single machine instruction is translated into (possibly several) operations in this Register Transfer Language.
\item \mbox{\hyperlink{coreclasses_classpcodeop}{Overview of Pcode\+Op}}
\end{DoxyItemize}
\item \mbox{\hyperlink{class_block_basic}{Block\+Basic}}
\begin{DoxyItemize}
\item A maximal sequence of p-\/code operations that always executes from the first \mbox{\hyperlink{class_pcode_op}{Pcode\+Op}} to the last.
\item \mbox{\hyperlink{coreclasses_classblockbasic}{Overview of Block\+Basic}}
\end{DoxyItemize}
\item \mbox{\hyperlink{class_funcdata}{Funcdata}}
\begin{DoxyItemize}
\item The root object holding all information about a function, including\+: the p-\/code syntax tree, prototype, and local symbol information.
\item \mbox{\hyperlink{coreclasses_classfuncdata}{Overview of Funcdata}}
\end{DoxyItemize}
\end{DoxyItemize}{\bfseries{Translation}} 
\begin{DoxyItemize}
\item \mbox{\hyperlink{coreclasses_classloadimage}{Load\+Image}}
\item \mbox{\hyperlink{coreclasses_classtranslate}{Translate}}
\end{DoxyItemize}

{\bfseries{Transformation}} 
\begin{DoxyItemize}
\item \mbox{\hyperlink{coreclasses_classaction}{Action}}
\item \mbox{\hyperlink{coreclasses_classrule}{Rule}}
\end{DoxyItemize}

{\bfseries{High-\/level}} {\bfseries{Representation}} 
\begin{DoxyItemize}
\item \mbox{\hyperlink{coreclasses_classdatatype}{Datatype}}
\item \mbox{\hyperlink{coreclasses_classtypefactory}{Type\+Factory}}
\item \mbox{\hyperlink{coreclasses_classhighvariable}{High\+Variable}}
\item \mbox{\hyperlink{coreclasses_classfuncproto}{Func\+Proto}}
\item \mbox{\hyperlink{coreclasses_classcallspecs}{Func\+Call\+Specs}}
\item \mbox{\hyperlink{coreclasses_classsymbol}{Symbol}}
\item \mbox{\hyperlink{coreclasses_classsymbolentry}{Symbol\+Entry}}
\item \mbox{\hyperlink{coreclasses_classscope}{Scope}}
\item \mbox{\hyperlink{coreclasses_classdatabase}{Database}}
\end{DoxyItemize}\hypertarget{coreclasses_classseqnum}{}\section{Overview of Seq\+Num}\label{coreclasses_classseqnum}
A sequence number is a form of extended address for multiple p-\/code operations that may be associated with the same address. There is a normal \mbox{\hyperlink{class_address}{Address}} field. There is a {\bfseries{time}} field which is a static value, determined when an operation is created, that guarantees the uniqueness of the \mbox{\hyperlink{class_seq_num}{Seq\+Num}}. There is also an {\bfseries{order}} field which preserves order information about operations within a basic block. This value may change if the syntax tree is manipulated.


\begin{DoxyCode}{0}
\DoxyCodeLine{\mbox{\hyperlink{class_address}{Address}} \& getAddr();          \textcolor{comment}{// get the Address field}}
\DoxyCodeLine{uintm     getTime();          \textcolor{comment}{// get the time field}}
\DoxyCodeLine{uintm     getOrder();         \textcolor{comment}{// get the order field}}
\end{DoxyCode}
\hypertarget{coreclasses_classpcodeop}{}\section{Overview of Pcode\+Op}\label{coreclasses_classpcodeop}
A single operation in the p-\/code language. It has, at most, one \mbox{\hyperlink{class_varnode}{Varnode}} output, and some number of \mbox{\hyperlink{class_varnode}{Varnode}} inputs. The inputs are operated on depending on the opcode of the instruction, producing the output.


\begin{DoxyCode}{0}
\DoxyCodeLine{\mbox{\hyperlink{opcodes_8hh_abeb7dfb0e9e2b3114e240a405d046ea7}{OpCode}}       code();             \textcolor{comment}{// get the opcode for this op}}
\DoxyCodeLine{\mbox{\hyperlink{class_address}{Address}} \&    getAddr();          \textcolor{comment}{// get Address of the associated processor instruction}}
\DoxyCodeLine{                                 \textcolor{comment}{// which generated this op.}}
\DoxyCodeLine{\mbox{\hyperlink{class_seq_num}{SeqNum}} \&     getSeqNum();        \textcolor{comment}{// get the full unique identifier for this op}}
\DoxyCodeLine{int4         numInput();         \textcolor{comment}{// get number of Varnode inputs to this op}}
\DoxyCodeLine{\mbox{\hyperlink{class_varnode}{Varnode}} *    getOut();           \textcolor{comment}{// get Varnode output}}
\DoxyCodeLine{\mbox{\hyperlink{class_varnode}{Varnode}} *    getIn(int4 i);      \textcolor{comment}{// get (one of the) Varnode inputs}}
\DoxyCodeLine{\mbox{\hyperlink{class_block_basic}{BlockBasic}} * getParent();        \textcolor{comment}{// get basic block containing this op}}
\DoxyCodeLine{\textcolor{keywordtype}{bool}         isDead();           \textcolor{comment}{// op may no longer be in syntax tree}}
\DoxyCodeLine{\textcolor{keywordtype}{bool}         isCall();           \textcolor{comment}{// various categories of op}}
\DoxyCodeLine{\textcolor{keywordtype}{bool}         isBranch();}
\DoxyCodeLine{\textcolor{keywordtype}{bool}         isBoolOutput();}
\end{DoxyCode}
\hypertarget{coreclasses_classblockbasic}{}\section{Overview of Block\+Basic}\label{coreclasses_classblockbasic}
A sequence of Pcode\+Ops with a single path of execution.


\begin{DoxyCode}{0}
\DoxyCodeLine{int4         sizeOut();         \textcolor{comment}{// get number of paths flowing out of this block}}
\DoxyCodeLine{int4         sizeIn();          \textcolor{comment}{// get number of paths flowing into this block}}
\DoxyCodeLine{\mbox{\hyperlink{class_block_basic}{BlockBasic}} *getIn(int4 i)       \textcolor{comment}{// get (one of the) blocks flowing into this}}
\DoxyCodeLine{\mbox{\hyperlink{class_block_basic}{BlockBasic}} *getOut(int4 i)      \textcolor{comment}{// get (one of the) blocks flowing out of this}}
\DoxyCodeLine{\mbox{\hyperlink{class_seq_num}{SeqNum}} \&    getStart();         \textcolor{comment}{// get SeqNum of first operation in block}}
\DoxyCodeLine{\mbox{\hyperlink{class_seq_num}{SeqNum}} \&    getStop();          \textcolor{comment}{// get SeqNum of last operation in block}}
\DoxyCodeLine{\mbox{\hyperlink{class_block_basic}{BlockBasic}} *getImmedDom();      \textcolor{comment}{// get immediate dominator block}}
\DoxyCodeLine{}
\DoxyCodeLine{iterator    beginOp();          \textcolor{comment}{// get iterator to first PcodeOp in block}}
\DoxyCodeLine{iterator    endOp();}
\end{DoxyCode}
\hypertarget{coreclasses_classfuncdata}{}\section{Overview of Funcdata}\label{coreclasses_classfuncdata}
This is a container for the sytax tree associated with a single {\itshape function} and all other function specific data. It has an associated start address, function prototype, and local scope.


\begin{DoxyCode}{0}
\DoxyCodeLine{\textcolor{keywordtype}{string} \&       getName();           \textcolor{comment}{// get name of function}}
\DoxyCodeLine{\mbox{\hyperlink{class_address}{Address}} \&      getAddress();        \textcolor{comment}{// get Address of function's entry point}}
\DoxyCodeLine{int4           numCalls();          \textcolor{comment}{// number of subfunctions called by this function}}
\DoxyCodeLine{\mbox{\hyperlink{class_func_call_specs}{FuncCallSpecs}} *getCallSpecs(int4 i); \textcolor{comment}{// get specs for one of the subfunctions}}
\DoxyCodeLine{\mbox{\hyperlink{class_block_graph}{BlockGraph}} \&   getBasicBlocks();    \textcolor{comment}{// get the collection of basic blocks}}
\DoxyCodeLine{}
\DoxyCodeLine{iterator       beginLoc(\mbox{\hyperlink{class_address}{Address}} \&);                     \textcolor{comment}{// Search for Varnodes in tree}}
\DoxyCodeLine{iterator       beginLoc(int4,\mbox{\hyperlink{class_address}{Address}} \&);                 \textcolor{comment}{// based on the Varnode's address}}
\DoxyCodeLine{iterator       beginLoc(int4,\mbox{\hyperlink{class_address}{Address}} \&,\mbox{\hyperlink{class_address}{Address}} \&,uintm);}
\DoxyCodeLine{iterator       beginDef(uint4,\mbox{\hyperlink{class_address}{Address}} \&);               \textcolor{comment}{// Search for Varnode based on the}}
\DoxyCodeLine{                                                        \textcolor{comment}{// address of its defining operation}}
\end{DoxyCode}
\hypertarget{coreclasses_classloadimage}{}\section{Load\+Image}\label{coreclasses_classloadimage}
\hypertarget{coreclasses_classaction}{}\section{Action}\label{coreclasses_classaction}
\hypertarget{coreclasses_classrule}{}\section{Rule}\label{coreclasses_classrule}
\hypertarget{coreclasses_classtranslate}{}\section{Translate}\label{coreclasses_classtranslate}
Decodes machine instructions and can produce p-\/code.


\begin{DoxyCode}{0}
\DoxyCodeLine{int4  oneInstruction(\mbox{\hyperlink{class_pcode_emit}{PcodeEmit}} \&,\mbox{\hyperlink{class_address}{Address}} \&) \textcolor{keyword}{const};   \textcolor{comment}{// produce pcode for one instruction}}
\DoxyCodeLine{\textcolor{keywordtype}{void} printAssembly(ostream \&,int4,\mbox{\hyperlink{class_address}{Address}} \&) \textcolor{keyword}{const};  \textcolor{comment}{// print the assembly for one instruction}}
\end{DoxyCode}
\hypertarget{coreclasses_classdatatype}{}\section{Datatype}\label{coreclasses_classdatatype}
Many objects have an associated \mbox{\hyperlink{class_datatype}{Datatype}}, including Varnodes, Symbols, and Func\+Protos. A \mbox{\hyperlink{class_datatype}{Datatype}} is built to resemble the type systems of common high-\/level languages like C or Java.


\begin{DoxyCode}{0}
\DoxyCodeLine{\mbox{\hyperlink{type_8hh_aef6429f2523cdf4d415ba04a0209e61f}{type\_metatype}} getMetatype();     \textcolor{comment}{// categorize type as VOID, UNKNOWN,}}
\DoxyCodeLine{                                 \textcolor{comment}{// INT, UINT, BOOL, CODE, FLOAT,}}
\DoxyCodeLine{                                 \textcolor{comment}{// PTR, ARRAY, STRUCT}}
\DoxyCodeLine{\textcolor{keywordtype}{string} \&      getName();         \textcolor{comment}{// get name of the type}}
\DoxyCodeLine{int4          getSize();         \textcolor{comment}{// get number of bytes encoding this type}}
\end{DoxyCode}


There are base types (in varying sizes) as returned by get\+Metatype.


\begin{DoxyCode}{0}
\DoxyCodeLine{\textcolor{keyword}{enum} \mbox{\hyperlink{type_8hh_aef6429f2523cdf4d415ba04a0209e61f}{type\_metatype}} \{}
\DoxyCodeLine{  \mbox{\hyperlink{type_8hh_aef6429f2523cdf4d415ba04a0209e61fa84de340fe64164ed0dcf473aad846961}{TYPE\_VOID}},       \textcolor{comment}{// void type}}
\DoxyCodeLine{  \mbox{\hyperlink{type_8hh_aef6429f2523cdf4d415ba04a0209e61fa64599dea7071bc28732936e0bfddff3e}{TYPE\_UNKNOWN}},    \textcolor{comment}{// unknown type}}
\DoxyCodeLine{  \mbox{\hyperlink{type_8hh_aef6429f2523cdf4d415ba04a0209e61fa4fcd4421533bb21c827865bdbac4fef9}{TYPE\_INT}},        \textcolor{comment}{// signed integer}}
\DoxyCodeLine{  \mbox{\hyperlink{type_8hh_aef6429f2523cdf4d415ba04a0209e61faba807ad7d41a3ac08fd4a71e9e013192}{TYPE\_UINT}},       \textcolor{comment}{// unsigned integer}}
\DoxyCodeLine{  \mbox{\hyperlink{type_8hh_aef6429f2523cdf4d415ba04a0209e61fa3e0f738756c5205d71c69d955603f46c}{TYPE\_BOOL}},       \textcolor{comment}{// boolean}}
\DoxyCodeLine{  \mbox{\hyperlink{type_8hh_aef6429f2523cdf4d415ba04a0209e61fa88922d1c65a64d2f39ebaba3411aa057}{TYPE\_CODE}},       \textcolor{comment}{// function data}}
\DoxyCodeLine{  \mbox{\hyperlink{type_8hh_aef6429f2523cdf4d415ba04a0209e61fa19a0a402d33a2e049e91ea9e37e90c2f}{TYPE\_FLOAT}},      \textcolor{comment}{// floating point}}
\DoxyCodeLine{\};}
\end{DoxyCode}


Then these can be used to build compound types, with pointer, array, and structure qualifiers.


\begin{DoxyCode}{0}
\DoxyCodeLine{\textcolor{keyword}{class }\mbox{\hyperlink{class_type_pointer}{TypePointer}} : \textcolor{keyword}{public} \mbox{\hyperlink{class_datatype}{Datatype}}  \{  \textcolor{comment}{// pointer to (some other type)}}
\DoxyCodeLine{  \mbox{\hyperlink{class_datatype}{Datatype}} *getBase();                  \textcolor{comment}{// get Datatype being pointed to}}
\DoxyCodeLine{\};}
\DoxyCodeLine{\textcolor{keyword}{class }\mbox{\hyperlink{class_type_array}{TypeArray}} : \textcolor{keyword}{public} \mbox{\hyperlink{class_datatype}{Datatype}} \{     \textcolor{comment}{// array of (some other type)}}
\DoxyCodeLine{  \mbox{\hyperlink{class_datatype}{Datatype}} *\mbox{\hyperlink{class_type_array_ab6e6a5e04271b6af1f5c85fee3d94f7e}{getBase}}();                  \textcolor{comment}{// get Datatype of array element}}
\DoxyCodeLine{\};}
\DoxyCodeLine{\textcolor{keyword}{class }\mbox{\hyperlink{class_type_struct}{TypeStruct}} : \textcolor{keyword}{public} \mbox{\hyperlink{class_datatype}{Datatype}} \{    \textcolor{comment}{// structure with fields of (some other types)}}
\DoxyCodeLine{  \mbox{\hyperlink{struct_type_field}{TypeField}} *\mbox{\hyperlink{class_type_struct_aebb450d9b8d2ad526b35ff135d5640a1}{getField}}(int4,int4,int4 *);   \textcolor{comment}{// get Datatype of a field}}
\DoxyCodeLine{\};}
\end{DoxyCode}
\hypertarget{coreclasses_classtypefactory}{}\section{Type\+Factory}\label{coreclasses_classtypefactory}
This is a container for Datatypes.


\begin{DoxyCode}{0}
\DoxyCodeLine{\mbox{\hyperlink{class_datatype}{Datatype}} *findByName(\textcolor{keywordtype}{string} \&);                 \textcolor{comment}{// find a Datatype by name}}
\DoxyCodeLine{\mbox{\hyperlink{class_datatype}{Datatype}} *getTypeVoid();                        \textcolor{comment}{// retrieve common types}}
\DoxyCodeLine{\mbox{\hyperlink{class_datatype}{Datatype}} *getTypeChar();}
\DoxyCodeLine{\mbox{\hyperlink{class_datatype}{Datatype}} *getBase(int4 size,\mbox{\hyperlink{type_8hh_aef6429f2523cdf4d415ba04a0209e61f}{type\_metatype}});}
\DoxyCodeLine{\mbox{\hyperlink{class_datatype}{Datatype}} *getTypePointer(int4,\mbox{\hyperlink{class_datatype}{Datatype}} *,uint4);   \textcolor{comment}{// get a pointer to another type}}
\DoxyCodeLine{\mbox{\hyperlink{class_datatype}{Datatype}} *getTypeArray(int4,\mbox{\hyperlink{class_datatype}{Datatype}} *);         \textcolor{comment}{// get an array of another type}}
\end{DoxyCode}
\hypertarget{coreclasses_classhighvariable}{}\section{High\+Variable}\label{coreclasses_classhighvariable}
A single high-\/level variable can move in and out of various memory locations and registers during the course of its lifetime. A \mbox{\hyperlink{class_high_variable}{High\+Variable}} encapsulates this concept. It is a collection of (low-\/level) Varnodes, all of which are used to store data for one high-\/level variable.


\begin{DoxyCode}{0}
\DoxyCodeLine{int4       numInstances();      \textcolor{comment}{// get number of different Varnodes associated}}
\DoxyCodeLine{                                \textcolor{comment}{// with this variable.}}
\DoxyCodeLine{\mbox{\hyperlink{class_varnode}{Varnode}} *  getInstance(int4);    \textcolor{comment}{// get (one of the) Varnodes associated with}}
\DoxyCodeLine{                                \textcolor{comment}{// this variable.}}
\DoxyCodeLine{\mbox{\hyperlink{class_datatype}{Datatype}} * getType();           \textcolor{comment}{// get Datatype of this variable}}
\DoxyCodeLine{\mbox{\hyperlink{class_symbol}{Symbol}} *   getSymbol();         \textcolor{comment}{// get Symbol associated with this variable}}
\end{DoxyCode}
\hypertarget{coreclasses_classfuncproto}{}\section{Func\+Proto}\label{coreclasses_classfuncproto}
\hypertarget{coreclasses_classcallspecs}{}\section{Func\+Call\+Specs}\label{coreclasses_classcallspecs}
\hypertarget{coreclasses_classsymbol}{}\section{Symbol}\label{coreclasses_classsymbol}
A particular symbol used for describing memory in the model. This behaves like a normal (high-\/level language) symbol. It lives in a scope, has a name, and has a \mbox{\hyperlink{class_datatype}{Datatype}}.


\begin{DoxyCode}{0}
\DoxyCodeLine{\textcolor{keywordtype}{string} \&      getName();        \textcolor{comment}{// get the name of the symbol}}
\DoxyCodeLine{\mbox{\hyperlink{class_datatype}{Datatype}} *    getType();        \textcolor{comment}{// get the Datatype of the symbol}}
\DoxyCodeLine{\mbox{\hyperlink{class_scope}{Scope}} *       getScope();       \textcolor{comment}{// get the scope containing the symbol}}
\DoxyCodeLine{\mbox{\hyperlink{class_symbol_entry}{SymbolEntry}} * getFirstWholeMap(); \textcolor{comment}{// get the (first) SymbolEntry associated}}
\DoxyCodeLine{                                \textcolor{comment}{// with this symbol}}
\end{DoxyCode}
\hypertarget{coreclasses_classsymbolentry}{}\section{Symbol\+Entry}\label{coreclasses_classsymbolentry}
This associates a memory location with a particular symbol, i.\+e. it {\itshape maps} the symbol to memory. Its, in theory, possible to have more than one \mbox{\hyperlink{class_symbol_entry}{Symbol\+Entry}} associated with a \mbox{\hyperlink{class_symbol}{Symbol}}.


\begin{DoxyCode}{0}
\DoxyCodeLine{\mbox{\hyperlink{class_address}{Address}} \&   getAddr();         \textcolor{comment}{// get Address of memory location}}
\DoxyCodeLine{int4        getSize();         \textcolor{comment}{// get size of memory location}}
\DoxyCodeLine{\mbox{\hyperlink{class_symbol}{Symbol}} *    getSymbol();       \textcolor{comment}{// get Symbol associated with location}}
\DoxyCodeLine{\mbox{\hyperlink{class_range_list}{RangeList}} \& getUseLimit();     \textcolor{comment}{// get range of code addresses for which}}
\DoxyCodeLine{                               \textcolor{comment}{// this mapping applies}}
\end{DoxyCode}
\hypertarget{coreclasses_classscope}{}\section{Scope}\label{coreclasses_classscope}
This is a container for symbols.


\begin{DoxyCode}{0}
\DoxyCodeLine{\mbox{\hyperlink{class_symbol_entry}{SymbolEntry}} *findAddr(\mbox{\hyperlink{class_address}{Address}} \&,\mbox{\hyperlink{class_address}{Address}} \&);          \textcolor{comment}{// find a Symbol by address}}
\DoxyCodeLine{\mbox{\hyperlink{class_symbol_entry}{SymbolEntry}} *findContainer(\mbox{\hyperlink{class_address}{Address}} \&,int4,\mbox{\hyperlink{class_address}{Address}} \&); \textcolor{comment}{// find containing symbol}}
\DoxyCodeLine{\mbox{\hyperlink{class_funcdata}{Funcdata}} *   findFunction(\mbox{\hyperlink{class_address}{Address}} \&);                \textcolor{comment}{// find a function by entry address}}
\DoxyCodeLine{\mbox{\hyperlink{class_symbol}{Symbol}} *     findByName(\textcolor{keywordtype}{string} \&);                   \textcolor{comment}{// find a Symbol by name}}
\DoxyCodeLine{\mbox{\hyperlink{class_symbol_entry}{SymbolEntry}} *queryByAddr(\mbox{\hyperlink{class_address}{Address}} \&,\mbox{\hyperlink{class_address}{Address}} \&);       \textcolor{comment}{// search for symbols across multiple scopes}}
\DoxyCodeLine{\mbox{\hyperlink{class_symbol_entry}{SymbolEntry}} *queryContainer(\mbox{\hyperlink{class_address}{Address}} \&,int4,\mbox{\hyperlink{class_address}{Address}} \&);}
\DoxyCodeLine{\mbox{\hyperlink{class_funcdata}{Funcdata}} *   queryFunction(\mbox{\hyperlink{class_address}{Address}} \&);}
\DoxyCodeLine{\mbox{\hyperlink{class_scope}{Scope}} *      discoverScope(\mbox{\hyperlink{class_address}{Address}} \&,int4,\mbox{\hyperlink{class_address}{Address}} \&); \textcolor{comment}{// discover scope of an address}}
\DoxyCodeLine{\textcolor{keywordtype}{string} \&     getName();                              \textcolor{comment}{// get name of scope}}
\DoxyCodeLine{\mbox{\hyperlink{class_scope}{Scope}} *      getParent();                            \textcolor{comment}{// get parent scope}}
\end{DoxyCode}
\hypertarget{coreclasses_classdatabase}{}\section{Database}\label{coreclasses_classdatabase}
This is the container for Scopes.


\begin{DoxyCode}{0}
\DoxyCodeLine{\mbox{\hyperlink{class_scope}{Scope}} *getGlobalScope();                \textcolor{comment}{// get the root/global scope}}
\DoxyCodeLine{\mbox{\hyperlink{class_scope}{Scope}} *resolveScope(\textcolor{keywordtype}{string} \&,\mbox{\hyperlink{class_scope}{Scope}} *);  \textcolor{comment}{// resolve a scope by name}}
\end{DoxyCode}
\hypertarget{coreclasses_classarchitecture}{}\section{Architecture}\label{coreclasses_classarchitecture}
This is the repository for all information about a particular processor and executable. It holds the symbol table, the processor translator, the load image, the type database, and the transform engine.


\begin{DoxyCode}{0}
\DoxyCodeLine{\textcolor{keyword}{class }\mbox{\hyperlink{class_architecture}{Architecture}} \{}
\DoxyCodeLine{  \mbox{\hyperlink{class_database}{Database}} *     \mbox{\hyperlink{class_architecture_ab426c9baa9013d9826041a4083e844ef}{symboltab}};   \textcolor{comment}{// the symbol table}}
\DoxyCodeLine{  \mbox{\hyperlink{class_translate}{Translate}} *    \mbox{\hyperlink{class_architecture_a81f2db17fdc609a7d1cedbc0a2eb7753}{translate}};   \textcolor{comment}{// the processor translator}}
\DoxyCodeLine{  \mbox{\hyperlink{class_load_image}{LoadImage}} *    \mbox{\hyperlink{class_architecture_a344b1348ed8f0bbe0d050e7038bb290d}{loader}};      \textcolor{comment}{// the executable loadimage}}
\DoxyCodeLine{  \mbox{\hyperlink{class_action_database}{ActionDatabase}} \mbox{\hyperlink{class_architecture_a4cbe1c5d170cb2a4d359765b97fef26f}{allacts}};     \textcolor{comment}{// transforms which can be performed}}
\DoxyCodeLine{  \mbox{\hyperlink{class_type_factory}{TypeFactory}} *  \mbox{\hyperlink{class_architecture_a8225ba7bd6ac802660e03ee793289dd1}{types}};       \textcolor{comment}{// the Datatype database}}
\DoxyCodeLine{\};}
\end{DoxyCode}
 